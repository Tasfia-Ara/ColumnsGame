\documentclass{article}

%% Page Margins %%
\usepackage{geometry}
\geometry{
    top = 0.75in,
    bottom = 0.75in,
    right = 0.75in,
    left = 0.75in,
}

\usepackage{amsmath}
\usepackage{graphicx}
\usepackage{parskip}

\title{Assembly Project: Dr Mario}

% TODO: Enter your name
\author{Elena Ding & Name 2}

\begin{document}
\maketitle

\section{Instruction and Summary}

\begin{enumerate}

    \item Which milestones were implemented? 
    Milestone 1, 2, 3

    \item How to view the game:
    % TODO: specify the pixes/unit, width and height of 
    %       your game, etc.  NOTE: list these details in
    %       the header of your breakout.asm file too!
    
    \begin{enumerate}

    \item Unit width/height in pixels: 8
    \item Display width/height in pixels: 256
    \item Base address for display: 0x10008000


    \end{enumerate}

    

\begin{figure}[ht!]
    \centering
    % \includegraphics[width=0.3\textwidth]{name.png}
    \caption{caption}
    \label{Instructions}
\end{figure}

\item Game Summary:
% TODO: Tell us a little about your game.
\begin{itemize}
\item Columns is like simplified tetris
\item 3 randomly coloured pixels are stacked on top of each other to form a rectangle which you can control with WASD, where ASD moves the rectangle and W shuffles the colours of the rectangle
\item Grid is 13 x 7
\item Match 3 colours in a row, column, or diagonal to clear those blocks and make space for new rectangles
\item Game is over when the rectangle hits the top and can't move downward.
\end{itemize}

    
\end{enumerate}

\section{Attribution Table}
% TODO: If you worked in partners, tell us who was 
%       responsible for which features. Some reweighting 
%       might be possible in cases where one group member
%       deserves extra credit for the work they put in.

\begin{center}
\begin{tabular}{|| c | c ||}
\hline
 Student 1 (Elena Ding (1011012841)) &  Student 2 (Name and student number) \\ 
 \hline
 Draw the first 3 gem column with random colours & Task\\
 \hline
 Implement shuffle logic with key W & Task\\
 \hline
 Allow player to quit game with key q & Task\\ 
 \hline
 When column moves into left or right side wall, keep it in same location & Task\\ 
 \hline
 When player tries to move a column left/right but another column is there, keep it in the same location & Task\\
 \hline
 End game if gems reach the top of the grid & Task\\
 \hline
 Task & Task\\  
 \hline
\end{tabular}
\end{center}

% TODO: Fill out the remainder of the document as you see 
%       fit, including as much detail as you think 
%       necessary to better understand your code. 
%       You can add extra sections and subsections to 
%       help us understand why you deserve marks for 
%       features that were more challenging than they
%       might initially seem.


\end{document}